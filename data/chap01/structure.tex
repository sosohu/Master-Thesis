
本文总共分为六章。

第一章首先介绍了Linux环境下主要图形系统结构、OpenGL和mesa3D的历史与现状,然后介绍了龙芯3A系列处理器的相关技术背景,最后交待了本文的主要工作和组织结构。

第二章主要介绍Mesa3D的主要结构和工作原理。重点对Mesa3D渲染管线实现原理和缓存区管理策略进行研究和分析,并对Mesa3D国内外相关的优化现状与优化成果进行总结。

第三章主要介绍Mesa3D缓存策略的优化。包括两方面的内容:第一是关于立即模式下数据缓存策略的原理分析,并结合龙芯3A实际平台特点进行优化;第二是关于显示列表模式下,数据缓存策略的原理分析,同样根据龙芯3A实际平台特点进行优化。

第四章主要介绍Mesa3D中的内存到显存数据传输的优化。研究和分析了原有Mesa3D中数据从内存传输到显存的传输链路,结合龙芯3A平台数据拷贝性能特点,提出了一个简化而又高效的数据传输链路,从而提高内存到显存的数据传输效率。

第五章主要是优化结果性能测试与分析部分。采用svPerfGL测试基准对前面提出的各个优化方法进行测试提升效果,并对结果进行分析。

第六章主要是对全文研究工作进行总结,并针对研究中遇到的一些问题,提出未来研究的方向。
