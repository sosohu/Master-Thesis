
本文的主要工作内容如下:

\begin{itemize}
\item{Mesa3D图形库实现原理分析} \\
本文梳理出了Mesa3D图形库的主要结构,并仔细分析了Mesa3D图形库的硬件加速流程,重点研究了显示列表、立即模式和顶点数组模式的实现机理。这些研究分析为寻找龙芯平台Mesa3D图形库优化方法提供理论基础。
\item{龙芯平台Mesa3D图形库性能瓶颈分析} \\
通过相关测试基准和性能分析工具定位出龙芯平台Mesa3D图形库的性能瓶颈,并且仔细分析产生这些瓶颈的软硬件原因,为提出相关优化方法做好了实验基础。
\item{龙芯平台Mesa3D图形库性能优化} \\
针对龙芯平台Mesa3D图形库的各个性能瓶颈,分别提出了CPU/GPU负载均衡、内存到显存数据传输优化和CPU端热点优化三种优化方法。
\item{龙芯平台Mesa3D图形库性能优化效果评测} \\
采用业界通用的测试基准对本文提出的优化方法测试优化效果并分析测试结果的一些潜在问题。
\end{itemize}



本文的研究工作均在龙芯3A系列软硬件平台上使用主流OpenGL性能测试集svPerfGL\footnote{svPerfGL是一个开源的OpenGL基准测试集合,它主要用来测量科学可视化应用的真实性能。}进行了测试和验证,其中立即模式性能提升约有10$\%$到40$\%$,显示列表性能提升约有xxx,内存向显存数据传输优化性能提升约有xxx。本文相关优化成果都已集成到应用到龙芯最新Mesa3D图形库中。
