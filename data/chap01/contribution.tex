
本文的主要工作内容如下:

\begin{itemize}
\item{\textbf{立即模式(Immediate Mode)\footnote{OpenGL传统的使用glBegin...glEnd方式制定绘制方式,在这两个函数对之间给出绘制的数据,这种方式称为立即模式。}下顶点数据传输过程研究与优化}}: \\
OpenGL的立即模式下,glBegin与glEnd之间的数据会逐次拷贝到顶点缓冲区中,每当顶点缓存去装载满数据后,这些数据会被传输到图形硬件设备(GPU)显存中,然后再继续装载未装载的数据。本文通过分析这个装载过程的性能瓶颈,提出了一种
\item{\textbf{显示列表(Display List)\footnote{OpenGL显示列表是由一组预先存储起来的留待以后调用的OpenGL函数语句组成的,当调用这张显示列表时就依次执行表中所列出的函数语句。}下CPU与GPU负载平衡的优化}}: \\
通过跟踪分析Mesa3D库代码发现在创建显示列表时候,会先预分配一块缓冲区,当数据溢出时会继续分配一块相同大小的缓冲区,重复此过程直到所有数据都存放到缓冲区中然后提交到显存设备中。这样在面临显示列表中包含大量顶点数据时,就需要分配很多次缓存区,这其中涉及到的数据准备和状态管理等操作都需要GPU来做,这便会导致CPU与GPU的工作负载不平衡,形成性能瓶颈。本文通过根据显示列表数据规模确定缓存区大小的优化方法来解决显示列表的性能瓶颈。
\item{\textbf{内存向显存数据传输优化}}: \\
由于Mesa3D中的顶点数据,属性数据等等都需要传输到显存中,所以内存到显存的数据传输效率就显得至关重要。本文研究分析了现有Mesa3D库内存到显存的数据传输过程,提出了一种简化的数据传输拷贝方式,提高了数据传输速度。
\end{itemize}



本文的研究工作均在龙芯3A系列软硬件平台上使用主流OpenGL性能测试集svPerfGL\footnote{svPerfGL是一个开源的OpenGL基准测试集合,它主要用来测量科学可视化应用的真实性能。}进行了测试和验证,其中立即模式性能提升约有xxx,显示列表性能提升约有xxx,内存向显存数据传输优化性能提升约有xxx。本文相关优化成果都已集成到应用到龙芯最新Mesa3D图形库中。
