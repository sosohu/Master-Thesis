%%% Local Variables: 
%%% mode: latex
%%% TeX-master: "../main"
%%% End: 

\chapter{OpenGL历史版本变更表 }
%\footnote{摘自OpenGL-维基百科}
\label{cha:OpenGL-History-Version}

%跨页表格
\begin{center} \tablecaption{OpenGL版本变更表 \label{tab:OpenGL-Version-History}} 
\tablefirsthead{
\rowcolor[gray]{0.8}
\multicolumn{1}{l}{\textbf{主要版本}} &
\multicolumn{1}{l}{\textbf{发布日期}} &
\multicolumn{1}{c}{\textbf{重要变更}} \\ }
\tablehead{\multicolumn{3}{c}{
\small 表 \ref{tab:OpenGL-Version-History} (续) } \\
\rowcolor[gray]{0.8}
\multicolumn{1}{l}{\textbf{主要版本}} &
\multicolumn{1}{l}{\textbf{发布日期}} &
\multicolumn{1}{c}{\textbf{重要变更}} \\ }
\tabletail{\bottomrule
\multicolumn{3}{c}{\small 接下页} \\}
\tablelasttail{\bottomrule}

\begin{supertabular}{p{2.cm}p{2.cm}p{11.cm}}
      1.1 & 1997.1 & \\
      1.2 & 1998.3 & \\
      1.2.1 & 1998.10 & \\
      1.3 & 2001.8 & \\
      1.4 & 2002.7 & \\
      1.5 & 2003.7 & \\
      2.0 & 2004.9 & \\
      2.1 & 2006.7 & \\
      3.0 & 2008.8 & \\
      3.1 & 2009.3 & \\
      3.2 & 2009.8 & \\
      3.3 & 2010.3 & \\
      4.0 & 2010.3 & \\
      4.1 & 2010.7 & \\
      4.2 & 2011.8 & 
支持的显卡: \\
& & Nvidia GeForce 400 series, Nvidia GeForce 500 series, Nvidia GeForce 600 series, ATI Radeon HD 5000 series, AMD Radeon HD 6000 Series, AMD Radeon HD 7000 Series \\
& & \tabitem 支持Shaders原子计数器和加载、存储、原子读-修改-写操作的单级纹理着色器。\\
& & \tabitem捕捉GPU-tessellated几何变换反馈的结果绘制的多个实例,使复杂的对象进行有效的重新定位和复制。\\
& & \tabitem支持修改任意子集的压缩纹理,而无需重新下载整个GPU的纹理,显著的性能改进。\\
& & \tabitem支持包装成一个单一的32位值显著降低内存存储和带宽的高效着色处理多个8位和16位值。\\
    %\toprule[1.5pt] 
      4.3 & 2012.8 & 支持的显卡: \\
& & NVIDIA GeForce400系列,NVIDIA GeForce500系列,NVIDIA GeForce600系列,NVIDIA GeForce700系列,ATI Radeon HD 5000系列,AMD Radeon HD6000系列,AMD Radeon HD7000系列,AMD Radeon HD8000系列 \\
& & \tabitem 围内充分利用GPU的并发计算着色器的图形管道 \\
& & \tabitem 暗器的存储缓冲器对象 \\
& & \tabitem 纹理参数查询 \\
& & \tabitem 作为标准功能的高质量的纹理压缩ETC2/EAC \\
& & \tabitem 完全兼容的OpenGL ES3.0的API \\
& & \tabitem 在应用程序开发过程中调试能力接收调试消息 \\
& & \tabitem 没有数据复制以不同的方式解释纹理的纹理意见 \\
& & \tabitem 增加了内存的安全性 \\
& & \tabitem 一个多应用的健壮性扩展 \\
      4.4 & 2013.7 & 支持的显卡: \\
& & DIA GeForce 400系列、NVIDIA GeForce 500系列、NVIDIA GeForce 600系列、NVIDIA GeForce 700系列,ATI Radeon HD 5000系列、AMD Radeon HD6000系列、AMD Radeon HD7000系列、AMD Radeon R9/R7 200系列 \\
& & \tabitem 缓冲器位置控制 \\
& & \tabitem 高效异步查询 \\
& & \tabitem 着色器可变布局 \\
& & \tabitem 高效多对象绑定 \\
& & \tabitem 精简化Direct3D应用的移植 \\
& & \tabitem 非绑定的纹理扩展 \\
& & \tabitem 稀疏纹理扩展 \\
      4.5 & 2014.8 & \\
\end{supertabular}
\end{center}

\chapter{Mesa3D历史版本变更表}
\label{cha:Mesa3D-History-Version}

\begin{center} \tablecaption{Mesa3D版本变更表(4.0之前版本不再支持) \label{tab:Mesa3D-Version-History}} 
\tablefirsthead{
\rowcolor[gray]{0.8}
\multicolumn{1}{l}{\textbf{主要版本}} &
\multicolumn{1}{l}{\textbf{发布日期}} &
\multicolumn{1}{c}{\textbf{支持OpenGL版本}} \\ }
\tablehead{\multicolumn{3}{c}{
\small 表 \ref{tab:OpenGL-Version-History} (续) } \\
\rowcolor[gray]{0.8}
\multicolumn{1}{l}{\textbf{主要版本}} &
\multicolumn{1}{l}{\textbf{发布日期}} &
\multicolumn{1}{c}{\textbf{支持OpenGL版本}} \\ }
\tabletail{\bottomrule
\multicolumn{3}{c}{\small 接下页} \\}
\tablelasttail{\bottomrule}

\begin{supertabular}{p{2.cm}p{2.cm}m{11.cm}}
      11.2 & 2016.3 & 4.2(Intel 3.3)\\
      11.1 & 2015.12 & 4.2(Intel 3.3)\\
      11.0 & 2015.9 & 4.2(Intel 3.3)\\
      10.6 & 2015.6 & 3.3\\
      10.5 & 2015.3 & 3.3\\
      10.4 & 2014.12 & 3.3\\
      10.3 & 2014.9 & 3.3\\
      10.2 & 2014.6 & 3.3\\
      10.1 & 2014.3 & 3.3\\
      10.0 & 2013.11 & 3.3\\
      9.0 & 2012.10 & 3.1\\
      8.0 & 2012.2 & 3.0\\
      7.0 & 2007.6 & 2.1\\
      6.0 & 2004.1 & 1.5\\
      5.0 & 2002.11 & 1.4\\
      4.0 & 2001.10 & 1.3\\
\end{supertabular}
\end{center}


\chapter{ATTR宏函数}
%\footnote{摘自OpenGL-维基百科}
\label{cha:attr}

\begin{lstlisting}
/**
 * This macro is used to implement all the glVertex, glColor, glTexCoord,
 * glVertexAttrib, etc functions.
 */
#define ATTR( A, N, T, V0, V1, V2, V3 )					\
do {									\
   struct vbo_exec_context *exec = &vbo_context(ctx)->exec;		\
									\
   if (unlikely(!(ctx->Driver.NeedFlush & FLUSH_UPDATE_CURRENT)))	\
      ctx->Driver.BeginVertices( ctx );					\
   									\
   if (unlikely(exec->vtx.active_sz[A] != N))				\
      vbo_exec_fixup_vertex(ctx, A, N);					\
   									\
   {									\
      GLfloat *dest = exec->vtx.attrptr[A];				\
      if (N>0) dest[0] = V0;						\
      if (N>1) dest[1] = V1;						\
      if (N>2) dest[2] = V2;						\
      if (N>3) dest[3] = V3;						\
      exec->vtx.attrtype[A] = T;                                        \
   }									\
									\
   if ((A) == 0) {							\
      /* This is a glVertex call */					\
      GLuint i;								\
									\
      for (i = 0; i < exec->vtx.vertex_size; i++)			\
	 exec->vtx.buffer_ptr[i] = exec->vtx.vertex[i];			\
									\
      exec->vtx.buffer_ptr += exec->vtx.vertex_size;			\    
									\
      /* Set FLUSH_STORED_VERTICES to indicate that there's now */	\
      /* something to draw (not just updating a color or texcoord).*/	\
      ctx->Driver.NeedFlush |= FLUSH_STORED_VERTICES;			\
									\
      if (++exec->vtx.vert_count >= exec->vtx.max_vert)			\
	 vbo_exec_vtx_wrap( exec );					\
   }									\
} while (0)
\end{lstlisting}


