
\chapter{总结与展望}

\section{本文工作总结}

随着国际形势的日益复杂,计算机工业体系国产化的进程在加速推进,拥有完全自主知识产权的龙芯系列处理器作为计算机工业体系的核心基石,承担着更大的责任与期望。为了提高龙芯系列处理器的核心竞争力,更好更快的推进计算机工业体系的国产化的步伐,龙芯系列处理器的工作性能和图形使用体验就显得尤为重要。本文正是在这样的一个大背景下,展开对龙芯平台上的Mesa3D图形库的研究与优化工作。通过阅读Mesa3D、libdrm和radeon相关驱动源代码,深入分析龙芯平台Mesa3D库的实现原理与性能瓶颈,并针对性的进行优化。本文所做的主要工作如下:

\begin{itemize}
\item{} Mesa3D图形库CPU与GPU端负载分析与优化,由于龙芯平台的特殊性,现有Mesa3D图形库在CPU端的工作任务过多,导致CPU端执行负载较大成为性能的瓶颈,本文通过定位负载不平衡的原因,并提出有效的解决办法达到龙芯平台上Mesa3D图形库CPU与GPU的执行负载相对均衡,实现整体性能的提升。
\item{}	Mesa3D图形库内存到显存的数据传输的优化。本文分析了原有内存到显存的数据传输方法,并结合相关研究成果,改进了Mesa3D图形库内存到显存的数据传输策略,并使用宽位访存指令做出更进一步的优化,达到数据传输性能的提升。
\item{} Mesa3D图形库CPU端热点的优化。本文定位和分析了Mesa3D图形库在CPU端存在的计算热点,然后采用分支减少、循环简化以及编译优化等常用程序优化手段进行优化,提高了CPU端的执行性能。
\end{itemize}

\section{下一步研究方向}

本文的优化方法虽然在一定程度上能够提高龙芯平台Mesa3D图形库的性能,但是仍然存在一些问题需要未来的研究工作给予解决,未来主要工作主要有以下几个方面:

\begin{itemize}
\item{\textbf{龙芯平台的通用性}}: \\
本文所有研究和优化工作都是在龙芯3A-780E开发板上进行的,而龙芯系列处理器种类繁多,不同处理器平台搭载的GPU情况都不相同,特别是龙芯最新推出的3A2000和3B2000系列处理器,访存带宽有着很大的改进和提升,所以可能会造成一些性能瓶颈的不同。这就还需要具体环境具体分析然后修改和适配本文提出的几种优化方法。

\item{\textbf{VxWorks操作系统上的Mesa3D库移植与优化}}: \\
龙芯系列产品,除了桌面系列产品以外还包括嵌入式系列产品,目前龙芯嵌入式产品主要搭载着VxWorks操作系统,而目前该系统还没有支持Mesa3D,为了丰富嵌入式操作系统的图形表现能力,未来还需要在VxWorks操作系统上进行Mesa3D的移植和优化。

\item{\textbf{Mesa3D图形库最新版本的移植和优化}}: \\
本文优化的Mesa3D图形库是10.1.6版本,该版本是目前龙芯平台上的最新Mesa3D版本,而Mesa3D目前最新推出了11.2.0版本,在架构和性能上有一些改进,所以未来可以将最新的Mesa3D版本移植到龙芯平台并做出相关优化。

\end{itemize}


